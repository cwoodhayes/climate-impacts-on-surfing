\documentclass[12pt, letterpaper]{article}
\usepackage[utf8]{inputenc}
\usepackage{graphicx}
\graphicspath{{img/}}
\usepackage{natbib}
\usepackage{gensymb}
\usepackage[margin=0.5in]{geometry}
%prevent LaTeX from shitting itself whenever there's a url with an underscore in it
% stolen from https://tex.stackexchange.com/questions/383678/underscore-in-bibtex-url
\usepackage[hidelinks]{hyperref}
\usepackage[T1]{fontenc}
\usepackage{wrapfig}

\title{Surf's Up: The Impact of Climate Change on Southern California Coastal Recreation}
\author{Conor Hayes \thanks{Written for Prof. Lowell Stott's course \textit{CORE103: Climate Challenges}, at the University of Southern California.}}
\date{April 2019}

% help out the natbib package a bit
\def\citeapos#1{\citeauthor{#1}'s (\citeyear{#1})}

\begin{document}
	\maketitle

	\begin{abstract}
		Surfing is a vital part of the economy and cultural identity of Southern California. But as the global climate undergoes significant changes over the next century, the future of the sport, and of the iconic surf breaks upon which it depends, is in some doubt. In order to provide an initial assay of surfing's future, this paper attempts to survey the effects of climate change on the most important environmental characteristics of surf breaks: swells, ocean floor topography, and wind forcing. Increases in mean wave height and power, as well as changing dominant swell direction, will make for short-term gains in surfability, but may also mean the degradation of the coastline and the conditions breaks are able to support. In addition, the wind conditions which both generate waves far offshore and interfere the waves as they break on the shore are expected to produce increasingly poor surfing conditions as the century progresses. Overall, expected changes in the climate will likely result in a short-term increase of surfability in the Southern California coast, but may spell trouble for the sport later on in the 21st century.
	\end{abstract}

	\section{Introduction}
	The effects of climate change on coastal environments is an area of vital interest to scientists and laypeople alike. However, coastal effects are also among the most difficult to study and model, due to both the complexity of the air-sea-land interface, and the relative absence of longitudinal time-series data on coastal waves and the subsurface sea dynamics that produce them. \citep{swells-deep-water-waves}. This means that the effects of climate change on the many stakeholders of the world's coastal regions are comparatively unstudied and unknown. \citep{swells-deep-water-waves}

	Among these many stakeholders are surfers and other water recreation-seekers. In the United States alone, surfing directly contributed \$7.48B to the economy in 2008, after increasing 14\% over the previous 4 years. \citep{surfing-value-to-society} In addition to this measurable benefit, surfing contributes to the economy indirectly as a key element of local and global culture. In California, surfing was officially declared the state sport in 2018 \citep{surfing-official-sport} and is a core element of the state's global image. Surfers themselves have long served as stewards of the coastal oceans and local environments, and have founded many of California's early coastal conservation movements. \citep{surfing-official-sport}.

	The composition and formation of waves suitable for surfing aren't often formally studied, but are fairly well known. According to \citet{surfing-science}, four characteristics are informally considered by surfers when selecting a suitable surfing wave (or "break"): 

	\begin{figure}[h]
		\centering
		\includegraphics[width=0.5\textwidth]{img/wave_elements.png}
		\caption{A viable surfing wave viewed from above. Its large $\alpha$ value means it breaks slowly and from left-to-right when viewed from shore.\citep{surfing-science}}
	\end{figure} 

	\begin{itemize}
		\item $H_B$: Wave height, measured from the crest to the trough of the wave.
		\item $\alpha$: Wave peel angle--the acute angle formed between the wave crest and the shore. This determines how quickly the wave will break. Greater values of $\alpha$ will make a slower-breaking wave.
		\item $B_I$: Wave breaking intensity. This is primarily determined by the seabed gradient orthogonal to the crest, and determines whether the wave will spill, collapse, or break (as is preferred by surfers).
		\item $S_L$: Wave section length. This describes the length of wave crest before a discontinuity in any of the three characteristics above--and thus wave behavior--occurs, and helps to determine the sort of maneuvers surfers can perform.
	\end{itemize}

	Surfing breaks are relatively rare, measured as a percentage of coastline, due in large part to the small range of acceptable values for the four characteristics listed above. Of primary importance to our discussion is that these wave characteristics are entirely determined by the natural environment which creates the surf break. In particular, the following three factors, when combined, control the characteristics listed above to make or break a surfing wave: \citep{surfing-science}
	\begin{itemize}
		\item Swell magnitude and direction (the dominant direction of propagation for incoming waves)
		\item Coastal ocean floor bathymetry
		\item Coastal wind direction and magnitude
	\end{itemize}
	The aim of this paper is to describe the predicted effects of climate change on the three factors above in Southern California, and thereby predict the future of surfing in the region. Southern California was chosen because of its status as a surfing mecca, because of its population density and economic importance, and because of the unusual availability of longitudinal water and wave data from the region. It is the hope of the author that the methodology described in this paper can be re-used in other regions once similar long-term observational data can be collected.

	\section{Methods}
	This paper takes a broad survey of the existing literature on the surf-influencing climate factors listed above, and reproduces their methods and results here, along with some original analysis. 

	\subsection{Swell magnitude and direction}
	In order to provide a broad understanding of the future changes in Southern California's surfable swells, this paper draws from \citet{swells-deep-water-waves} and \citet{swells-aleutian-low}. In the first, \citeauthor{swells-deep-water-waves} attempt to describe and explain existing wave dynamics within the Southern California Bight (SCB), which is the roughly concave region of coastline extending from San Diego to Point Conception. The authors assimilate existing data from a variety of sources: 5 hindcast studies of the northern Pacific Ocean, NOAA buoy data from the last 30 years, systematic beach-based visual estimates, and energy-frequency spectrum data from wave measurement arrays near the shore. They then treat the data to a statistical analysis in search of correlations with known climatological factors--in particular, the Aleutian Low, the El Ni\~{n}o Southern Oscillation (ENSO), and the Pacific Decadal Oscillation (PDO)--as well as correlations with specific swell directions and topographical features of the coast. In addition to empirical data collection and analysis, the authors use a modeling approach called SWAN to determine the total energy delivered to the coastline from a variety of swell events. SWAN uses typical atmospheric and wind conditions during particular storm events as initial conditions for a wave propogation model ending with ocean waves breaking upon the shore. \citep{swells-deep-water-waves}

	\begin{figure}[h]
		\centering
		\includegraphics[width=.5\textwidth]{img/swells_scb_directions.png}
		\caption{Orientations for the six major wave types occurring in the SCB. (1) Aleutian low, (2) Pineapple Express, (3) northwest swell, (4) tropical storm, (5) Southern Hemisphere swell, and (6) local sea breeze.\citep{swells-deep-water-waves}}
		\label{fig:swell_direction}
	\end{figure}

	In the second article, \citeauthor{swells-aleutian-low} use a CMIP5 model to predict the response of the Aleutian Low to global climate change. The authors analyze three sets of simulations: a preindustrial control simulation with atmospheric $CO_2$ fixed at 280ppm; a modern control simulation in which atmospheric measurements over the last century are used as as forcings for the coupled model; and a projection run which follows RCP8.5 from 2000 until 2050. They informed the second simulation using 20th-century observational data of the North Pacific Index (NPI), an area-weighted average of sea level barometric pressure (SLP) over the region bounded by 30\degree –65\degree N and 160\degree E–140\degree W. The authors then confirmed the results of the CMIP5 modeling using a CAM3.1-RGO model, which is a coupled atmospheric-ocean model previously used to successfully model ENSO and tropical oceanic processes. \citep{swells-aleutian-low}

	\subsection{Coastal ocean floor bathymetry}
	Predicted changes in ocean floor topography and depth are referenced from the aforementioned \citet{swells-deep-water-waves} as well as \citet{bath-cosmos}. The latter predicts long-term shoreline evolution in Southern California with climate change using CoSMoS, a physics-based numerical modeling tool that uses knowledge of local bathymetry, topography, and hydrodynamics as well as global climate forcings. Its initial shoreline data is extracted from LIDAR surveys, as well as USGS National and Regional Assessments. The CoSMoS model also incorporates fluvial discharge measurements, as well as modeled currents and waves (using SWAN, the same model used by \citet{swells-deep-water-waves}), to predict the effect on sediment transport. Its eventual output is a coastline shape as a function of time, over the next 100 years. 

	\begin{figure}[h]
		\centering
		\includegraphics[width=1\textwidth]{img/cosmos_SLR_method.png}
		\caption{CoSMos calculates flood level as the mean sea level added to the local SLR, tides, seasonal effects, storm surges, and movement of individual waves. \citep{bath-cosmos}}
	\end{figure} 

	\subsection {Coastal wind direction and magnitude}
	This paper's analysis of wind direction and magnitude is informed by \citet{winds-coastal} and \citet{winds-santa-ana}, who describe the two primary sources of westerly and easterly winds, respectively, in Southern California. The former article attempts to verify the "Bakun Hypothesis" made in 1990, which predicted that increasing GHG concentrations would intensify upwelling-favorable winds in eastern ocean-boundary current systems (ECBS's) like the coast of California. A meta-analysis of 22 separate studies of wind trends in these regions between 1990 and 2012 was then conducted, and the resulting database was statistically analyzed to test the null hypothesis of upwelling winds increasing or decreasing with equal probility. While the article's meta-analysis also covers other eastern boundary current systems off the Chilean, Spanish, and South African coasts, the section of their article dealing with the California system is of primary interest to this paper. \citep{winds-coastal}

	The latter article \citep{winds-santa-ana} attempts to predict the changes in the Santa Ana winds (SAW), which are the strongest easterly winds of Southern California, and which help drive its wildfire season. The authors developed a statistical model to downscale low-resolution global climate model (GCM) pressure gradient data into the higher-resolution data needed to describe and predict the SAW. They trained their model on measurements of the SAW taken over the last 50 years, then selected 8 GCM's from the CMIP5 set for their ability to predict key aspects of the California climate. For each of these GCM's, they then generated predictions for the next 100 years using forcings from RCP8.5. \citep{winds-santa-ana}

	\begin{figure}[h]
		\centering
		\includegraphics[width=.5\textwidth]{img/winds_saw_direction.png}
		\caption{\citep{winds-santa-ana}}
	\end{figure}

	\section{Results}
	\subsection{Swell magnitude and direction}

	\begin{figure}[h]
		\centering
		\includegraphics[width=.4\textwidth]{img/swells_aleutian_vs_pinapple.png}
		\caption{SWAN model-derived maps of wave heights within the SCB during (a) Aleutian low wave-source conditions and (b) Pineapple Express wave conditions. Models are initialized with deep-water wave conditions for typical Aleutian low storm source (HK = 5 m, Ts = 15 s, $\alpha$ = 305\degree) and for typical Pineapple Express storm (Hs = 5 m, Ts = 15 s, $\alpha$ = 270\degree), respectively. Note that this $\alpha$ is the angle-of-attack of the waves relative to true-north, measured clockwise, and not the $\alpha$ described above as the angle-of-attack of the waves relative to the beach normal vector.\citep{swells-deep-water-waves}}
		\label{fig:aleutian_vs_pineapple}
	\end{figure}

	Of the wave types listed in Figure~\ref{fig:swell_direction}, the Aleutian low (labelled as 1 and 2) generates by far the most significantly portion of the average energy flux incident on the SCB. However, the Aleutian low itself shifts southward during El Ni\~no years, causing the "Pineapple Express" waves listed in the figure as (2). Although the total incident energy from the Pinapple Express waves and the "normal" northerly Aleutian low waves is similar, the amount of wave energy that actually reaches the coast from the Pinapple Express swell is approx. 320\% that of the normal Aleutian low. \citep{swells-deep-water-waves} As is visually apparent in Figure~\ref{fig:aleutian_vs_pineapple}, the SWAN model indicates that this is due to the shielding effect of the Channel Islands, combined with the slight southward tilt of the SCB.

	The Aleutian low itself is predicted to increase in intensity over the next 50 years, according to \citeauthor{swells-aleutian-low}. Figure \ref{fig:aleutian_npi_change}.b depicts the anomaly between the 2050 low and the current low, displaying a multimodel ensemble mean (MMEM) loss of 140Pa across the region in the RCP8.5 scenario, which is 62\% larger than the unforced internal variability of the low. Twelve of the 22 models used show a southward movement of the center of the low (defined roughly as its point of minimum pressure), while the remaining 10 show the center moving north, indicating an uncertain change in its latitudinal position. \citep{swells-aleutian-low}

	\begin{figure}[h]
		\centering
		\includegraphics[width=.8\textwidth]{img/swells_aleutian_npi_change.jpeg}
		\caption{
		(a) NPI changes in 22 CMIP5 models between the 50-yr RCP8.5 run and historical run. Asterisks (error bar) denote the 95\% confidence level (interval) based on a two-tailed Student’s t test. (b) The MMEM of winter SLP in the RCP8.5 run. A cross marks the minimum SLP of 995.5 hPa and dashed contours denote the isobars of 997.0 and 999.0 hPa. (c) The estimated internal variability of winter SLP based on the CMIP5 preindustrial control experiments. (d) The MMEM of winter SLP changes between the 50-yr RCP8.5 run and historical run. Stippling indicates regions where the MMEM change is greater than two standard deviations of internal variability and where at least 85\% (19/22) of the models agree on the sign of change. Hatching indicates regions where the MMEM change is in the range of one to two standard deviations of internal variability. The rest are regions where the MMEM change is less than one standard deviation of internal variability. \citep{swells-aleutian-low}}
		\label{fig:aleutian_npi_change}
	\end{figure}

	\subsection{Coastal ocean floor bathymetry}
	\citeauthor{bath-cosmos} predict increased retreat rates for all modeled stretches of the SCB coastline under RCP8.5 forcings. Large swaths of the Santa Monica region in particular are expected to retreat at approx. 1 m/yr, indicating significant changes to both underwater and abovewater topography. Under a 20mm/yr SLR scenario, portions of the Camp Pendleton coast (which includes the iconic beach breaks of San Onofre) are expected to retreat at over 4 m/yr.
	Significant changes will also be occurring in California's bays and lowland coastlines. Figure \ref{fig:slr_los_angeles} shows the flood map of Los Angeles after 150cm of SLR, which depicts much of the LA harbor and nearby areas well inside the flooding zone. \citep{bath-cosmos}

	\begin{figure}[h]
		\centering
		\includegraphics[width=.4\textwidth]{img/bath_slr_los_angeles.png}
		\caption{Predicted flood map for a 100-yr storm event over Los Angeles, given a local mean sea level rise of 150cm. The light blue overlay on top of land areas represents the regions left underwater for some duration of the flooding, while the pure blue represents the current location of the ocean without storm conditions.\citep{bath-cosmos}}
		\label{fig:slr_los_angeles}
	\end{figure}

	\subsection {Coastal wind direction and magnitude}
	\citeauthor{winds-santa-ana} predict a gradual decrease in SAW intensity over the next 50 years, with a particularly pronounced attenuation of SAW frequency and intensity in the tails of the wind season (-32\% MMEM reduction in March and October). December will see the least attenuation, according to nearly all models, meaning that peak SAW season will be earlier in the year. The results also indicated a strong correlation between reduction in SAW and a drop in the pressure gradient force (PGF) pushing outward from the western side of the Rocky Mountains on SAW days. Generally, easterly, onshore winds in Southern California will reduce significantly over the course of the 21st century under RCP8.5. \citep{winds-santa-ana}

	On the other hand, the meta-analysis conducted by \citeauthor{winds-coastal} indicates that Southern California's westerly winds, stemming from the California EBCS, are already increasing under climate change. Both model and observational data agree that for the California system, upwelling (warm-season) average wind speeds will increase by nearly 5mph over the course of the next half-century during the warm season, and a less significant increase overall. \citep{winds-coastal}

	\section {Discussion and Conclusion}
	With regards to swell magnitude and direction, the strengthened Aleutian low coupled with the increased frequency and magnitude of the ENSO \citep{bath-el-nino} will lead to stronger, more frequent Pinapple Express waves into the Southern California Bight, increasing mean wave height and the frequency of large swells--a positive result in the short-term for surfers. However, the increased net energy flux brings along with it a significantly increased coastal erosion rate, as modeled by \citeauthor{bath-cosmos} and referenced by \citeauthor{swells-deep-water-waves}. According to \citeauthor{swells-deep-water-waves}, "[Due primarily to the PDO], we expect to see a systematic decrease in beach sediment with time, resulting in exposure of the coastal bedrock (platforms and sea cliffs) to wave attack." This gradual loss may result in the degradation of existing breaks over the course of the 21st century. 

	Meanwhile, the wind forecast is quite unfavorable for the surf industry as climate change progresses. Offshore winds--in this case, coming from the west--are a detriment to surfing waves, forcing them to collapse or tumble rather than to break, rendering them undesirable for all but the most basic surfing. Onshore winds--here, from the east--serve the opposite function, in moderation--propping waves up and slowing them down for a more accessible, workable ride. \citep{surfing-science} The expected decrease of the Santa Ana winds (easterlies) coupled with the increase of the offshore EBCS winds will make for consistently worse surf over the course of this century. An additional consequence of increased upwelling winds in the California eastern boundary current system will likely result in colder waters, especially during the warm season\citep{winds-coastal}--a truly unwanted consequence in the already chilly eastern Pacific.

	Overall, the short-term prognosis for surfing in SoCal is bright, but long-term the costs will outweigh the benefits. It may be worth investigating methods to protect the most historic and beloved breaks in California over the next half-century, before they are degraded beyond recovery and their benefits to our culture and economy are lost.

	\pagebreak
	\bibliographystyle{plainnat}
	\bibliography{final_paper_cwhayes}
\end{document}	